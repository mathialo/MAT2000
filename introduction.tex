\chapter{Introduction}
\todo[inline]{Wrtie introduction}

In this report we will first give a general introduction to the basic concepts of compressed sensing, and then look at how these techniques can be applied to the face recognition problem. 


\section{Preliminaries and notation}
In this section we will introduce some of the necessary notation and concepts for this report. 

Throughout the report we will denote vectors by boldface lower case letters, and matrices by boldface upper case letters. For a vector $ \x $, we will by $ x_{i} $ refer to the $ i $'th element in the vector. Similarly, for a matrix $ \A $, we will by $ \a_{i} $ refer to the $ i $'th  column of $ \A $. 

Sets will be denoted by italic upper case letters, and the cardinality of a set $ S $ is denoted as $ \abs{S} $. The complement of a set $ S $ will be written as $ \overline{S} $. 

\subsection{Norms}
We begin with the definition of norms. 

\begin{definition} \label{def:norm}
	Let $ V $ be a vector space over $ \K $.  A norm $ \norm{\cdot} $ on $ V $ is a function $ \norm{\cdot}\colon V \to\R $ such that
	\begin{subdef}
		\item $ \norm{\v} \geq 0 $ for all $ \v \in V $ with $ \norm{\v} = \0 $ if and only if $ \v = \0 $
		\item $ \norm{c\v} = \abs{c}\norm{\v} $ for all $ \v \in V $ and $ c \in \K $
		\item $ \norm{\u + \v} \leq \norm{\u} + \norm{\v} $ for all $ \u,  \v \in V $
	\end{subdef}
\end{definition}

This definition is quite broad, and very general. In this report we will be working more closely with a family of norms called the $ \ell_{p} $ norms:

\begin{definition} \label{def:lpnorm}
	Let $ p \geq 1 $ and $ p \in \R $. The $ \ell_{p} $ norm $ \norm{\cdot}_{p} $ is defined as
	\[
		\norm{\v}_{p} = \left( \sum_{i=1}^{n} \abs{v_{i}}^{p} \right)^{\dfrac{1}{p}}
	\]
\end{definition}

We will not prove here that the $ \ell_{p} $-norms actually are norms (ie, that $ \norm{\cdot}_{p} $ fulfills the axioms in \cref{def:norm}), but this can be proven for all $ p\geq 1 $. We observe that for $ p=1 $ we get the Manhattan norm, and for $ p=2 $ we get euclidean norm. If we let $ p\to\infty $ we arrive at the supremum norm. Even though \cref{def:lpnorm} does not allow for $ p < 1 $, if we let $ p\to0 $, accept that $ 0^{0} = 0 $, and ignore the $ 1/p $-exponent, we get what is sometimes called the $ \ell_{0} $ norm:

\begin{definition} \label{def:l0norm}
	The $ \ell_{0} $ norm $ \norm{\cdot}_{0} $ is the number of non-zero entries in $ \v $.
\end{definition}

It is worth noting that the $ \ell_{0} $ norm is strictly speaking not a norm, since $ \norm{\cdot}_{0} $ does not fulfill axiom \textit{(ii)} of \cref{def:norm}. Despite this, it is customary to refer to $ \norm{\cdot}_{0} $ as the $ \ell_{0} $ norm.

\subsection{Support and sparsity}

\begin{definition} \label{def:support}
	Let $ \v \in \Cx^{N} $. The support $ S $ of $ \v $ is defined as the index set of its non-zero entries, that is:
	\[
		\supp \v = \set{j \in \indexset{N} \st v_{j} \neq 0}
	\]
\end{definition}

The notion of support yields a new formulation of \cref{def:l0norm}: The $ \ell_{0} $ norm is simply the cardinality of the support: $ \norm{\v}_{0} = \abs{\supp \v} $.

For a vector $ \v\in\Cx^{N} $ and a set $ S \subset \indexset{N} $, we denote by $ \v_{S} $ either the subvector in $ \Cx^{\abs{S}} $ consisting of the entries in $ \v $ indexed by $ S $, that is:
\begin{equation}
	\label{eq:vSalt1}
	(\v_{S})_{i} = v_{i} \for i \in S
\end{equation}
Or the vector in $ \Cx^{N} $ which coincides with $ \v $ on the indexes in $ S $, and is zero otherwise, that is:
\begin{equation}
	\label{eq:vSalt2}
	(\v_{S})_{i} = \twopartdef{v_{i}}{\text{if } i \in S}{0}{\text{Otherwise}}
\end{equation}
It should always be clear from context which of these is used. Similarly, for a matrix $ \A\in\R^{m \times n} $ we will by $ \A_{S} \in \R^{m \times \abs{S}} $ refer to the matrix consisting of the columns of $ \A $ indexed by $ S $.

The final concept we will introduce is the notion of \textit{sparsity}:

\begin{definition} A vector $ \v \in \Cx^{N} $ is said to be $ s $-sparse if it has no more than $ s $ non-zero entries. That is, $ \norm{\v}_{0} \leq s $
\end{definition}

